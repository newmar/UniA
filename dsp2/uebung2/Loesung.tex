\documentclass[12pt]{scrreprt}
\usepackage{acronym}
\usepackage{amsmath}
\usepackage{float}
\usepackage{amssymb}
\usepackage[utf8]{inputenc}
\usepackage[ngerman]{babel}
\usepackage{url}
\usepackage{tabularx}
\usepackage{longtable}
\usepackage{graphicx}
\usepackage{subfigure}
\usepackage{units}
\usepackage{setspace}
\usepackage{tikz}
\usepackage{pgfplots}
\title{Lawinensimulation}
\author{B.Sc. Markus Neuerburg}
\date{16.05.2012}

\begin{document}
\setlength{\topmargin}{0cm}
\parindent 0pt
\section*{DSP II Übungsblatt 2}
\subsection*{Aufgabe 1}
\underline{\textbf{Gegeben:}}\newline
\hspace*{5mm}y[n]=6x[n]-5x[n-1]+x[n-2]\newline
\newline
\underline{\textbf{Gesucht:}}\newline
\hspace*{5mm}Pol- und Nullstellen\newline
\newline
\hspace*{5mm}$H[z]=\frac{6z^2 - 5z + 1}{z^2}$
\hspace*{5mm}\underline{Polstellen:}\newline
\hspace*{5mm}0 doppelt\newline
\hspace*{5mm}\underline{Nullstellen:}\newline
\hspace*{5mm}$1-5z+6z^2=0$\newline
\hspace*{5mm}$\Rightarrow$ $x_{1,2}=\frac{5 \pm \sqrt{5^2 - 24}}{12}$\newline
\hspace*{5mm}$x_{1,2}=\frac{5 \pm 1}{12}$\newline
\hspace*{5mm}$x_{1,2}=\{0.5, \frac{1}{3}\}$

\subsection*{Aufgabe 2}
\underline{\textbf{Gegeben:}}\newline
\hspace*{5mm}$H(z)=4(1-z^{-1})(1+z^{-1})(1+0.8z^{-1})$\newline
\hspace*{5mm}$H(z)=4((1+Z^{-1}-Z^{-1}-z^{-2})(1+0.8z^{-1}))$\newline
\hspace*{5mm}$H(z)=4((1-z^{-2})(1+0.8z^{-1}))$\newline
\hspace*{5mm}$H(z)=4(1+0.8z^{-1} - z^{-2} -0.8z^{-3} )$\newline
\hspace*{5mm}$H(z)=4+3.2z^{-1} - 4z^{-2} - 3.2z^{-3}$\newline
\underline{\textbf{Gesucht:}}\newline
\hspace*{5mm}$h[n]=-3.2x[n-3] - 4x[n-2] + 3.2x[n-1] + 4x[n]$

\subsection*{Aufgabe 3}
Ausgangssignal $y[n]$:\newline
\hspace*{5mm}$y[n]=x[n]*h[n] = X[z]H[z] = Y[z]$\newline
\hspace*{5mm}$X[z]=z^{-1} - z^{-2} + z^{-3} - z^{-4}$\newline
\hspace*{5mm}$H[z]=1 + 2z^{-1} - 3z^{-2} + 4z^{-3}$\newline
\hspace*{5mm}$Y[z]=X[z]H[z]$


\hspace*{5mm}$y[n]=\sum_{m=0}^{N-1}x[m]h[n-m]$\newline
\hspace*{5mm}$y[n]=x[0]h[n-0]+x[1]h[n-1]+x[2]h[n-2]+x[3]h[n-3]$\newline
\newpage
z-Transformierte $Y(z)$:\newline
\hspace*{5mm}$Y(z)=X(z)H(z)$\newline
\hspace*{5mm}$Y(z)=(\sum_{n=-\infty}^{\infty}x[n]z^{-n})(\sum_{n=-\infty}^{\infty}h[n]z^{-n})$

\subsection*{Aufgabe 4}

\subsection*{Aufgabe 5}
\subsubsection*{a)}
Impulsantwort $h[n]$:\newline
\hspace*{5mm}$b_k={3, +2, -3}$\newline
\hspace*{5mm}$h[n]=\sum_{k=0}^{M}b_k\delta[n-k]$\newline
\hspace*{5mm}$h[n]=3\delta[n-0] + 2\delta[n-1] - 3\delta[n-2]$\newline
%TODO SKIZZE

\subsubsection*{b)}
\underline{\textbf{Gegeben:}}\newline
\hspace*{5mm}$x[n]=3e^{j(0.4\pi n - \frac{\pi}{2})}$ für alle n\newline
\hspace*{5mm}
%TODO Weiter

\subsection*{Aufgabe 6}
\underline{\textbf{Gegeben:}}\newline
\hspace*{5mm}Frequenzgang: $H(\hat{\omega})=2jsin(\frac{\hat{\omega}}{2})e^{-j\frac{\hat{\omega}}{2}}$\newline
\underline{\textbf{Gesucht:}}\newline
\hspace*{5mm}Impulsantwort $h[n]$ und Differenzengleichung $y[n]$\newline
%TODO Weiter

\subsection*{Aufgabe 7}
\hspace*{5mm}$h[n]=\sum_{m=0}^{N-1}h_2[m]h_1[n-m]$\newline
\newline
\begin{tabular}{r|cccccc}
$n$ 				&  0  &  1  &  2  & 3   &  4  & 5 \\\hline
$h_1[n]$			&  1  &  2  &  3  & 4   &     &   \\
$h_2[n]$			& -1  &  1  & -1  &     &     &   \\\hline
$h_2[0]h_1[n-0]$	& -1  & -2  & -3  & -4  &     &   \\
$h_2[1]h_1[n-1]$	&     &  1  &  2  &  3  &  4  &   \\
$h_2[2]h_1[n-2]$	&     &     & -1  & -2  & -3  & -4\\\hline
$y[n]$              & -1  & -1  & -2  & -3  &  1  & -4\\
\end{tabular}
%TODO Frequenzgang

\subsection*{Aufgabe 8}
\subsubsection*{a)}
$b_k={1, 2, 1}$\newline
$H(\hat{\omega})=1-2e^{-j\hat{\omega}}+e^{-2j\hat{\omega}}=1-2cos(\hat{\omega})+ 
j sin(\hat{\omega}) + e^{-2j\hat{\omega}} + cos(\hat{\omega}) - 2j sin(\hat{\omega})$\newline
$H(\hat{\omega})=1 - cos(\hat{\omega}) - j sin(\hat{\omega})$\newline
$\Re\{H(\hat{\omega})\}=(1 - cos(\hat{\omega}))$, $\Im\{H(\hat{\omega})\}=-sin(\hat{\omega})$\newline
$\vert H(\hat{\omega}) \vert = \left[(1-cos(\hat{\omega}))^2 - sin^2(\hat{\omega})\right]^{\frac{1}{2}}$\newline
$\vert H(\hat{\omega}) \vert = \left[1-2cos(\hat{\omega}) + cos(\hat{\omega})^2 - \frac{(1-cos(2\hat{\omega}))}{2}\right]^{\frac{1}{2}}$\newline
Laut Skript:\newline
$\vert H(\hat{\omega}) \vert = \left[ 2 (1-cos(\hat{\omega}))\right]^\frac{1}{2} = 2 \vert sin(\frac{\hat{\omega}}{2})\vert$\newline
$\angle H(\hat{\omega}) = tan^{-1}\left(\frac{sin(\hat{\omega})}{1-cos(\hat{\omega})}\right)$\newline
\newline

\subsubsection*{b)}
\begin{tikzpicture}
    \begin{axis}[domain=-3:3,legend pos=outer north east]
    \addplot[blue, mark=none]{2*abs{sin(x/2)}}; 
    \legend{$2\vert sin(x/2) \vert$}
    \end{axis}
\end{tikzpicture}
\subsubsection*{c)}
Es handelt sich um einen Hochpass.\newline

\subsection*{Aufgabe 9}
$h[n]=b_0(a_1)^n u[n] + b_1(a_7)^{n-7} u[n-7]=
\begin{cases}
0							& \text{für } n<0\\
b_0 						& \text{für } n=0\\
(b_0 + b_7 a_1^{-7})(a_1)^n	& \text{sonst}
\end{cases}
$\newline
\begin{tikzpicture}
    \begin{axis}[domain=0:10,legend pos=outer north east]
    \addplot[blue, mark=none]{(1/156250)*(0.5)^x}; 
    \legend{$(5 0.5^{-7})(0.5)^n$}
    \end{axis}
\end{tikzpicture}

\subsection*{Aufgabe 10}
\begin{tikzpicture}
\draw (0, 0) circle (1);
\draw (0, 1.5) -- (0, -1.5);
\draw (1.5, 0) -- (-1.5, 0);
\draw [green](1, 0) circle (0.1);
\draw [red](1.25, 0) circle (0.1);
\end{tikzpicture}\newline
Polstelle: rot ($Z=1,25$)\newline
Nullstelle: grün ($Z=1$)\newline
Das System ist zwar kausal aber nicht stabil
\end{document}