\documentclass[12pt]{scrreprt}
\usepackage{acronym}
\usepackage{amsmath}
\usepackage{float}
\usepackage{amssymb}
\usepackage[utf8]{inputenc}
\usepackage[ngerman]{babel}
\usepackage{url}
\usepackage{tabularx}
\usepackage{longtable}
\usepackage{graphicx}
\usepackage{subfigure}
\usepackage{units}
\usepackage{setspace}
\usepackage{tikz}
\title{Lawinensimulation}
\author{B.Sc. Markus Neuerburg}
\date{16.05.2012}

\begin{document}
\setlength{\topmargin}{0cm}
\parindent 0pt
\section*{DSP II Übungsblatt 1}
\subsection*{Aufgabe 1.1}
Bei einem Energiesignal handelt es sich um ein Signal bei dem die Signalenergie bzw. mittlere Signalleistung endlich ist. Das bedeutet, dass das Signal zwar Periodisch sein kann aber begrenzt sein muss. Zum Beispiel durch einen Dämpfungsfaktor. Ein Beispiel sei hier das Schwingen eines Gewichts an einer realistischen Feder.\newline
\newline
Bei einem Leistungssignal handelt es sich um ein Signal bei dem die Signalenergie bzw. die mittlere Signalleistung unendlich sind.
Ein Beispiel hierfür ist hier stochastisches Rausche, ein Kosinus- oder Sinus-Signal.\newline
\section*{Aufgabe 1.2}

\subsection*{Aufgabe 1.6}
\subsubsection*{a)}
\begin{equation}
	x(t)=A cos(\omega_0 t + \phi) = A cos(\omega_0(t-t_1))
\end{equation}
Gegeben:\newline
$T=0,5 sec,$\newline
$t_1=0,1 sec$\newline
Aus obiger Gleichung ist ersichtlich, dass $\phi=-t_1 \omega_0$ sein muss mit $\omega_0= 2 \pi f$ und $f=\frac{1}{T}$ folgt: $\phi=-t_1 \frac{2 \pi}{T} = -0,1sec \frac{2 \pi}{0,5sec} = -1,26 rad  \frac{180}{\pi} = -72,2^\circ$
\subsubsection*{b)}
\begin{equation}
	x(t)=Re \lbrace e^{j 6 \pi (t + 0,2)} \rbrace
\end{equation}
Gesucht:\newline
Zeitverschiebung $t_1$\newline
Aus obiger Gleichung folgt: $Re \lbrace e^{j 6 \pi (t + 0,2)} \rbrace = cos[6\pi(t + 0,2)] = cos[6\pi(t - (-0,2))] \newline \Rightarrow t_1 = -0,2$
\subsection*{Aufgabe 1.7}
\subsubsection*{a)}
\begin{equation}
	x_1(t)=7cos(\omega_0 t - \frac{3\pi}{4}, x_2(t)=10cos(\omega_0 t - \frac{2\pi}{3}
\end{equation}
Gesucht: \newline
Das überlagerte Signal $x_3(t) = x_1(t) + x_2(t)$ in Kosinusform $Acos(\omega_0 + \phi)$
\end{document}