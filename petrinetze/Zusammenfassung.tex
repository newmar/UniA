\documentclass[12pt]{scrreprt}
\usepackage{acronym}
\usepackage{amsmath}
\usepackage{amssymb}

\usepackage[ngerman]{babel}

\usepackage{hyperref}

\usepackage{breakurl}

\usepackage{cite}
\usepackage{color}

\usepackage{float}

\usepackage{graphicx}



\usepackage[utf8]{inputenc}

\usepackage{listings}
\usepackage{longtable}

\usepackage{url}

\usepackage{rotating}

\usepackage{setspace}
\usepackage{stmaryrd}
\usepackage{subfigure}

\usepackage{tabularx}
\usepackage{tabto}
\usepackage{threeparttable}
\usepackage{tikz}

\usepackage{units}

\usetikzlibrary{arrows, shapes, decorations, automata, backgrounds, petri}
\urlstyle{same}
\definecolor{dkgreen}{rgb}{0,0.6,0}
\definecolor{gray}{rgb}{0.5,0.5,0.5}
\definecolor{mauve}{rgb}{0.58,0,0.82}
\def\TReg{\textsuperscript{\textregistered}}
\def\TCop{\textsuperscript{\textcopyright}}
\def\TTra{\textsuperscript{\texttrademark}}

\def\colCOL#1{\multicolumn{1}{>{\columncolor{green!30}}c}{#1}}
\def\colCell#1#2{\multicolumn{1}{>{\columncolor{#1}}c}{#2}} 


\title{Zusammenfassung Petrinetze}
\author{B. Sc. Markus Neuerburg}
\date{\today}

\begin{document}
\setlength{\topmargin}{0cm}
\parindent 0pt
\onehalfspacing

\tableofcontents
\newpage
\chapter*{Abkürzungsverzeichnis}
\begin{acronym}[HBCI]
	\acro{t*}[$\mathrm{T^*}$]{akzeptiertes Alphabet}
	\acro{a*}[$\mathrm{A^*}$]{akzeptierte Buchstaben}
	\acro{es}[$\mathrm{\lambda}$]{leere Schaltfolge}
	\acro{<p}[$<_p$]{lexikographische Ordnung}
	\acro{mu}[$\mu$]{Multimenge}
\end{acronym}

\chapter{Grundbegriffe}
\begin{itemize}
	\item $t \in T:$						\tabto{4cm} Transition aus der Menge aller Transitionen
	\item $s \in S:$						\tabto{4cm} Stelle aus der Menge aller Stellen
	\item $(x, y) \in F:$					\tabto{4cm} Kannte aus der Menge aller Flussrelationen (Kanten)
	\item $W:$								\tabto{4cm} Gewichtungs-Funktion
	\item $W(x, y):$						\tabto{4cm} Kanntengewicht (Gewicht auf den Pfeilen)
	\item $^\bullet x=\{y \mid (y,x) \in F \}$	\tabto{4cm} Der Vorbereich von x\newline
	Sprich: Vorbereich von x ist y mit der Eigenschaft: Kannte von y nach x ist Element aller Flussrelationen (Pfeile)
	\item $x^\bullet =\{y \mid (x, y) \in F\}$	\tabto{4cm} Der NAchbereich von x\newline
	Sprich: Nachbereich von x ist y mit der Eigenschaft: Kannte von x nach y ist Element aller Flussrelationen (Pfeile)
	\item $M:S\mapsto \mathbb{N}$			\tabto{4cm} Markierung\newline
	Eine Markierung M ist eine Menge von Stellen abgebildet auf $\mathbb{N}$
	\item $\left[ M_N \right>$ ist die Menge der in Netz N erreichbaren Markierungen (Erreichbarkeitsmenge)
\end{itemize}

\chapter{Definitionen}
\section{Definition 2.2 (aktivierte Transition)}
$t \in T$ ist aktiviert unter Markierung $M$, $M\left[t\right>$, falls $\forall s \in S:W(s,t) \leq M(s)$\newline
	Sprich: Transition $t$ ist aktiviert unter Markierung $M$, falls für alle Stellen aus der Menge $S$ gilt, dass das Kanntengewicht der Kannte von s nach t kleiner oder gleich Anzahl der Marken auf Stelle $s$ ($M(s)$) ist.
\section{Definition 2.3 (Schaltfolge)}
Sei $w \in$ \ac{t*} : $M\left[w\right>$ bzw. $M\left[w\right>M'$ falls:
\begin{itemize}
	\item $w =$ \ac{es} (und $M=M'$)
	\item $w = w't$ mit $t \in T$, $M\left[w'\right>M''\left[t'\right>$ (und $M''\left[t\right>M'$)
\end{itemize}
$FS(N)=\{w \in$ \ac{aa}$ \mid M_N \left[w\right>\}$ Menge der Schaltfolgen von N (firing sequence)\newline
$w \in T^\omega$ unendliche Schaltfolge (falls alle endlichen Präfixe von w Schaltfolgen sind)

\section{Definition 2.21}
\subsection{tot}
$t$ heißt tot unter $M$, falls $\forall M' \in \left[M\right> : \neg M'\left[t\right>$\newline
$t$ heißt tot, falls $t$ tot unter $M_N$.\newline
$M$ heißt tot, falls alle Transitionen tot unter $M$ sind.
\newpage
\subsection{lebendig}
$t$ heißt lebendig unter $M$, falls $t$ unter keiner von $M$ erreichbaren Markierung $M'$ tot ist:\newline
$\forall M' \in \left[M\right>\exists M'' \in \left[M'\right> :  M'' \left[t\right>$\newline
$M$ heißt lebendig, wenn alle $t$ unter $M$ lebendig sind.\newline
$t$ heißt lebendig, falls $t$ lebendig unter $M_N$.\newline
$N$ heißt lebendig, falls alle Transitionen lebendig sind.

Eine Transition $t$ eines Netzes $N$ ist
\begin{itemize}
	\item 0-lebendig, wenn sie niemals schalten kann.
	\item 1-lebendig, wenn sie mindestens einmal schalten kann.
	\item 2-lebendig, wenn es für jedes $n \in N$ eine Schaltfolge gibt, in der $t$ mindestens n-mal schaltet.
	\item 3-lebendig, wenn es eine unendliche Schaltfolge gibt, in der $t$ unendlich oft schaltet.
	\item 4-lebendig, wenn es von jeder erreichbaren Markierung aus eine Schaltfolge mit $t$ gibt.
\end{itemize}

\subsection{Identifikation von S-Invarianten}
S-Invarianten Sind nützlich um Situationen zu untersuchen bei denen es zu Konflikten kommen kann. z.B. reader writer Prozess. Hier ist zu beachten, dass lesen und schreiben nicht gleichzeitig geschehen dürfen. Dass dies sichergestellt werden kann dürfen zu keinem Zeitpunkt die Marken auf der Stelle für die Zugriffsrechte ausrechen, dass nach dem Starten eines Schreibprozess sofort der Leseprozess geschalten werden kann. Das kann nur der Fall sein, wenn jede Marke auch wirklich verbraucht wird und immer nur soviele Marken auf die Stellen gelegt werden wie auch verbraucht werden. Gleiches gilt auch für den Leseprozess. Hier dürfen jedoch mehrere Leseprozesse hintereinander gestartet werden. Auch hier gilt: Alle Leseprozesse müssen erst beendet werden, dass ein Schreibprozess gestartet werden darf.\newline

S-Invarianten:
\begin{itemize}
\item Inzidenzmatrix aufstellen $s1..sn, t1...tm$
\item gleiche Stellen in den s-Zeilen finden die Null ergeben $\rightarrow$ $y1...yn$
\end{itemize}

\subsubsection*{Satz 3.4}
Ist keine Transition in $N$ tot (oder $N$ gar lebendig) und $y$ ein ganzzahliger Vektor mit $y^{T} * M = y^{T} * M_{N}$ für alle $M \in [M_{N}>$, dann ist $y$ eine S-Invariante.\newline

\subsection*{Identifikation von T-Invarianten}
- Spalten der Inzidenzmatrix

\section*{Definition 3.2 (S-Invarianten überdeckt)}
$N$ heisst von S-Invarianten überdeckt, wenn $N$ eine positive S-Invariante besitzt. (positiv: $\forall y(s) > 0$)

\section*{Satz 3.5 (beschränkt, sicher)}
Ist $s \in S$ und $y$ eine nicht-negative S-Invariante mit $y(s) > 0$, dann ist $s$ - auch bei geänderter Anfangsmarkierung $M_{0}$ - beschränkt. Ist $y^{T} * M_{0} = 1$, so ist s sicher unter $M_{0}$.

\subsubsection*{Satz 3.10 (T-Invarianten überdeckt)}
wenn $N$ bei einer Anfangsmarkierung M lebendig und beschränkt ist, dann ist N von T-Invarianten überdekt.

\section*{Siphon}
$R \subseteq S$ ist ein Siphon (deadlock, Tube), falls $^\bullet R \subseteq R^\bullet$

\section*{Falle}
$R \subseteq S$ ist eine Falle, falls $R^\bullet \subseteq ^\bullet R$

\section*{Satz von Commoner}
Sei N ein EFC-Netz ohne isolierte Stellen.\newline
N ist lebendig $\Leftrightarrow$ Jeder nichtleere Siphon enthält eine markierte Falle.

\chapter{Fairness}
\section{Schwache Fairness}
Eine Schaltfolge ist schwach fair bezüglich einer Transition $t$, wenn sie
\begin{itemize}
	\item endlich ist und in einer Markierung endet, die t nicht aktiviert
	\item unendlich ist und t unendlich oft deaktiviert ist
	\item t unendlich oft enthält
\end{itemize}

\section{Strakte Fairness}
Eine Schaltfolge ist stark fair bezüglich einer Transition t, wenn sie
\begin{itemize}
	\item endlich ist un in einer Markierung endet, die t nicht aktiviert,
	\item unendlich ist und t nur endlich oft aktiviert ist
	\item t unendlich oft enthält
\end{itemize}

Eine Schaltfolge ist schwach fair bzw. stark fair, wenn sie für jede Transition des Netzes schwach fair bzw. stark fair ist.

Andere Definition:\newline
schach fair:\newline
\tabto{0.5cm}Transition, die unendlich oft aktiviert ist, schaltet auch\newline
stark fair:\newline
\tabto{0.5cm}Transition, die unendlich oft immer wieder aktiviert ist, schaltet auch

\chapter{Graphen}
\section{Erreichbarkeitsgraph}
\subsection{Definition Erreichbarkeit}
$M'$ heißt erreichbar von Markierung $M$, sprich ist $M' \in [M>$, falls $w \in T*$ existiert mit $M [w> M'$.

\section{Überdeckbarkeitsgraph}

\chapter{Lemmata}
\section{Dicksons Lemma (5.3)}
Sei ${(M_i)}_{i \in \mathbb{N}}$ eine Folge von erweiterten Markierungen. Dann existiert eine (unendliche) schwach monotone Teilfolge $({M_i}_j)_{j \in \mathbb{N}}$, d.h. $\forall j \in \mathbb{N}$ gilt $i_j < i_{j+1}$ und ${M_i}_j \le {M_i}_{j+1}$.

\section{Königs Lemma (5.5)}
Sei (V,E) ein unendlicher, lokal endlicher gerichteter Graph, so dass zu einem $v_0 \in V$ für alle $v \in V$ ein $v_0 v$-Weg existiert. Dann existiert ein unendlicher Weg $v_0 v_1 ...$

\section{Lemma 6.3}
N habe keine isolierten Stellen, und $R \not = \emptyset$ sei ein Siphon. Dann ist $R^\bullet \not = \emptyset$

\section{Lemma 6.9}
\begin{enumerate}
	\item Die Vereinigung von Siphons (Fallen) ist ein(e) Siphon (Falle).
	\item Jedes $R \subseteq S$ enthält eine größte (d.h. eindeutige maximale) Falle
\end{enumerate}

\section{Lemma 6.10}
\ac{<p} ist wohlgegründet, d.h. es gibt keine unendliche echt absteigende Kette $(M_1 >_p M_2 >_p M_3 ...)$

\section{Lemma 6.11}
Sei N ein EFC-Netz (extended free choice), $R \subseteq S$ und Q die maximale (d.h. größte) Falle in R. Dann gibt es eine Ordnung P von $R \ Q$, so dass für alle Markierungen M, unter denen Q unmarkiert ist, gilt:\newline
$[(\exists t \in R^\bullet : M\left[t\right>) \vee (R$ ist Siphon $\wedge \exists t \in R^\bullet$ : t ist nicht tot unter M)] $\Rightarrow [ \exists M' \in \left[M\right> : M' $\ac{<p}$M]$ und Q ist unmarkiert unter M'.

\section{Pumping Lemma}


\chapter{Übungen}

\section{Übung 1}
\subsection*{Aufgabe 1}
\begin{figure}[H]
\begin{tikzpicture}[node distance=1.3cm, >=stealth', bend angle=45, auto]
  \tikzstyle{place}=[circle, thick, draw=blue!75, fill=blue!20, minimum size=6mm]
  \tikzstyle{red place}=[place, draw=red!75, fill=red!20]
  \tikzstyle{transition}=[rectangle, thick, draw=black!75, fill=black!20, minimum size=4mm]
  \tikzstyle{every label}=[red]

	\begin{scope}
		\node[place, tokens=1] 	at(0,0)		(s1){};
		\node[place]			at(0,-2)	(s2){};
		\node[place]			at(2.1,0)	(s3){};
		\node[place, tokens=1]	at(2.1,-2)	(s4){};
		\node[place, tokens=1]	at(4.1,0)	(s5){};
		\node[place]			at(4.1,-2)	(s6){};
		\node[place]			at(6.1,0)	(s7){};
		\node[place, tokens=1]	at(6.1,-2)	(s8){};
		
		\node[transition]	at(-1,-1)	(t1){}
			edge[pre, bend left]			(s1)
			edge[post,bend right]			(s2);
		\node[transition]	at(1.1,-1)	(t2){}
			edge[pre, bend left]			(s2)
			edge[post,bend right]			(s1)
			edge[pre, bend right]			(s4)
			edge[post,bend left]			(s3);
		\node[transition]	at(3.1,-1)	(t3){}
			edge[pre, bend right]			(s3)
			edge[post,bend left]			(s4)
			edge[pre, bend left]			(s5)
			edge[post,bend right]			(s6);
		\node[transition]	at(5.1,-1)	(t4){}
			edge[pre, bend left]			(s6)
			edge[post,bend right]			(s5)
			edge[pre, bend right]			(s8)
			edge[post,bend left]			(s7);
		\node[transition]	at(7.1,-1)	(t5){}
			edge[pre, bend right]			(s7)
			edge[post,bend left]			(s8);
	\end{scope}
\end{tikzpicture}
\caption{Lösung 1}
\end{figure}
\begin{figure}[H]
\begin{tikzpicture}[node distance=1.3cm, >=stealth', bend angle=45, auto]
  \tikzstyle{place}=[circle, thick, draw=blue!75, fill=blue!20, minimum size=6mm]
  \tikzstyle{red place}=[place, draw=red!75, fill=red!20]
  \tikzstyle{transition}=[rectangle, thick, draw=black!75, fill=black!20, minimum size=4mm]
  \tikzstyle{every label}=[red]

	\begin{scope}
		\node[place, tokens=1] 	at(0,0)		(s1){};
		\node[place]			at(0,-2)	(s2){};
		\node[place]			at(2.1,0)	(s3){};
		\node[place, tokens=2]	at(2.1,-2)	(s4){};
		\node[place, tokens=1]	at(4.1,0)	(s5){};
		\node[place]			at(4.1,-2)	(s6){};
		
		\node[transition]	at(-1,-1)	(t1){}
			edge[pre, bend left]			(s1)
			edge[post,bend right]			(s2);
		\node[transition]	at(1.1,-1)	(t2){}
			edge[pre, bend left]			(s2)
			edge[post,bend right]			(s1)
			edge[pre, bend right]			(s4)
			edge[post,bend left]			(s3);
		\node[transition]	at(3.1,-1)	(t3){}
			edge[pre, bend right]			(s3)
			edge[post,bend left]			(s4)
			edge[pre, bend left]			(s5)
			edge[post,bend right]			(s6);
		\node[transition]	at(5.1,-1)	(t4){}
			edge[pre, bend left]			(s6)
			edge[post,bend right]			(s5);
	\end{scope}
\end{tikzpicture}
\caption{Lösung 2}
\end{figure}
\subsection*{Aufgabe 2}
\subsubsection*{a)}
$n=0:$ $M(1,0,0) \in M_N$\newline
$n=1:$ $M(1,0,1)$\newline
Erreichbar mit:\newline
$M_N\left[a\right>M'\left[b\right>M(1,0,1)$ = $M_N\left[(ab)\right>M(1,0,1)$ $w=(a,b)$\newline
$n=n:$ $M_N\left[(ab)^n\right>M(1,0,n)$
\newpage
\subsubsection*{b)}
%TODO Bild
Erreichbare Markierungen sind:\newline
$(0,0,0) , (1,0,0) , (1,0,1) , (1,0,n) , (1,0,(n+1))$\newline
$(1,0,0)\left[(ab)\right>(1,0,1)$\newline
$(1,0,0)\left[(ab)^n\right>(1,0,n)$\newline
$(1,0,0)\left[(ab)^{n+1}\right>(1,0,(n+1))$\newline
$(1,0,0)\left[(ab)\right>(1,0,1)\left[(c\right>(0,0,(n-1))$\newline
$(1,0,0)\left[(ab)^nc\right>(0,0,n)$\newline

\subsection*{d}
$R={(1,0,n), (0,1,(n+1)), (0,0,n)\mid n \in \mathbb{N}} \subseteq \left[M_N\right>$\newline
$R \supseteq \left[M_N\right>$\newline
Behauptung: $M_N\left[w\right>M\Rightarrow M \in R$ Induktion über w\newline
\newline
Beweis:
\begin{itemize}
	\item $w=\lambda: M=M_N \in R$\textcolor{green}{\checkmark}
	\item $w=w't: M_N\left[w'\right>M'\left[t\right>M$ Nach Induktion $M'\in R$\textcolor{green}{\checkmark}
	\item $M'=(1,0,n): t=a:M={(0,1,(n+1))} \in R$ \textcolor{green}{\checkmark}
\end{itemize}
$t=c \wedge n>=1: M={(0,0,(n-1))} \in R$ \textcolor{green}{\checkmark}\newline
$M'={(0,1,(n+1))}: t=b: M={(1,0,(n+1))} \in R$\textcolor{green}{\checkmark}\newline
$M'={(0,0,n)}: \neg \exists t$  \textcolor{red}{$\lightning$}

\newpage

\subsection*{Aufgabe 3}
Alle Markierungen:
\begin{figure}[H]
\begin{tikzpicture}
	\node at (0,0) 			(m1){1,1,0,0,0};
	\node at (0,-1.5)		(m2){0,0,1,0,0}
		edge [pre] node[right] {s+} (m1);
	\node at (0,-3)			(m3){0,0,0,1,1}
		edge [pre] node[right] {e+} (m2);
	\node at (-2, -4.5)		(m4){1,0,0,1,0}
		edge [pre] node[left] {s-}(m3)
		edge [post, bend left] node[left] {e-} (m1);
	\node at (2, -4.5)		(m5){0,1,0,0,1}
		edge [pre] node[right] {e-} (m3)
		edge [post, bend right] node[right] {s-} (m1);

\end{tikzpicture}
\end{figure}
Alle Codes:
\begin{figure}[H]
\begin{tikzpicture}
	\node at (0,0) 			(m1){0,0};
	\node at (0,-1.5)		(m2){0,1}
		edge [pre] node[right] {s+} (m1);
	\node at (0,-3)			(m3){1,1}
		edge [pre] node[right] {e+} (m2);
	\node at (-2, -4.5)		(m4){0,1}
		edge [pre] node[left] {s-}(m3)
		edge [post, bend left] node[left] {e-} (m1)
		edge [post, color=red, bend left] (m2);
	\node at (2, -4.5)		(m5){1,0}
		edge [pre] node[right] {e-} (m3)
		edge [post, bend right] node[right] {s-} (m1);
\end{tikzpicture}
\end{figure}
Es besteht ein Konflikt. Zu vermeiden wäre der Konflikt mit einer zusätzlichen Stelle im Code.\newline
Der STG ist jedoch konsistent, da nie s+, e+, s- oder e- zweimal hintereinander oder s+ und s- bzw e+ und e- sofort hintereinander ausgeführt werden.
\newpage

\section{Übung 2}
\subsection*{Aufgabe 2}
\subsubsection*{a)}
Behauptung:\newline
$M_{N'} = M_N + \delta M$ 	\tabto{5cm} $\Rightarrow FS(N) <= FS(N')$\newline
Beweis:\newline
$M_N \left[w\right>M'$ 	\tabto{5cm} $\Leftarrow w \in FS(N)$\newline
$M_N + \delta M \left[w\right>M_N' + \delta M$ \tabto{5cm} $\Rightarrow w \in FS(N')$\newline
Nach 2.7.\newline
$M_{N'} \left[w\right>M_N' + \delta M$

\subsubsection*{b)}
$s'=S-\left\lbrace s_0\right\rbrace, W' = W \mid _{S-\left\lbrace s_0 \right\rbrace}, M_{N'} = M_N \mid  _{S-\left\lbrace s_0\right\rbrace}$\newline
$M_N\left[w\right>M \Rightarrow M_{N'}\left[w\right>M' \wedge M' = M \mid _S-\left\lbrace s_0 \right\rbrace$\newline
Induktionsanfang:\newline
$w=\lambda:$\tabto{2cm}$M_{N} \left[w\right> M \Rightarrow M_N = M$\newline
			\tabto{2cm}$M_{N'} \left[w\right> M' \Rightarrow M_{N'} = M \mid _S-\left\lbrace s_0 \right\rbrace$\newline
Induktionsschritt:\newline
$w=w't:$\tabto{2cm}$M_N \left[w'\right>M''$\newline
		\tabto{2cm}$M_{N' \left[w'\right>}M''' \wedge M''' = M'' \mid _S-\left\lbrace s_0 \right\rbrace$ nach Induktion\newline
Die Markierungen sind die Selben. Das 2.Netz hat lediglich eine Stelle weniger\newline
Zu zeigen: \tabto{2cm} $M''\left[t\right>M \Rightarrow M'''\left[t\right>M' \wedge M' = M \mid _S-\left\lbrace s_0 \right\rbrace$

\newpage

\section{Übung 3}
\subsection*{Aufgabe 1}
Schalten von 0 Transitionen:\newline
$M_N \left[\lambda\right> M$ \tabto{4cm} $M_N = M = {1,2,1}$\newline
Schalten von 1 Transition:\newline
Schalten von a b oder c:\newline
${(1,2,1)} \left[a\right> {(0,1,1)}$\newline
${(1,2,1)} \left[b\right> {(1,1,0)}$\newline
${(1,2,1)} \left[c\right> {(1,2,0)}$\newline
${(1,2,1)} \left[d\right> {(1,1,2)}$\newline
Da es sich um eine Halbordnung handelt muss jedes Element einer Markierung mit den anderen Elementen der anderen Markierungen verglichen werden. Daraus resultiert, dass es sich bei $(1,1,2)$ und $(1,2,1)$ um maximale Markierungen handelt.

\section{Übung 4}
\subsection*{Aufgabe 1}
$N_1:$	\tabto{2cm} ist lebendig\newline
		\tabto{2cm} ist verklemmungsfrei\newline
$N_2:$	\tabto{2cm} ist nicht tot\newline
		\tabto{2cm} ist nicht verklemmungsfrei\newline
$N_3:$	\tabto{2cm} ist lebendig\newline
		\tabto{2cm} ist nicht verklemmungsfrei\newline
$N_4:$	\tabto{2cm} ist lebendig\newline
		\tabto{2cm} ist nicht verklemmungsfrei\newline
$N_5:$	\tabto{2cm} ist tot\newline
		\tabto{2cm} ist nicht verklemmungsfrei\newline

\subsection*{Aufgabe 2}
Es gibt keine lebendige Markierung

\subsection*{Aufgabe 4}
$r_1(r_2 e_2 L_2)^\omega$\newline
schwach fair, da $e_1$ nicht fair behandelt wird\newline
$e_2$ ist zwar unendlich oft aktiviert aber auch unendlich oft deaktiviert.\newline
stark fair:\newline
$r_1$ und $l_1$ sind nie aktiviert ($\Rightarrow$ können sich nicht beschweren, tun nichts zur Sache)\newline
S-Invariante $\Rightarrow$ $r_1$ leer $l_1$ immer leer\newline
$r_2 e_2 l_2$: $e_1$ wird nicht stark fair behandelt.,

\newpage

\section{Übung 5}
\subsection*{Aufgabe 1}
\subsubsection*{a)}
Suchen der S-Invarianten:\newline
Inzidenzmatrix:\newline
$\bordermatrix{ %
		&e_1&e_2&r_1&r_2&l_1&l_2&g_1&g_2\cr
req_1	& -1&  0& +1&  0&  0&  0&  0&  0\cr
req_2	&  0& -1&  0& +1&  0&  0&  0&  0\cr
c_1		& +1&  0&  0&  0& -1&  0&  0&  0\cr
c_2		&  0& +1&  0&  0&  0& -1&  0&  0\cr
tok_1	& -1&  0&  0&  0&  0& +1& -1& +1\cr
tok_2	&  0& -1&  0&  0& +1&  0& +1& -1\cr
nc_1	&  0&  0& -1&  0& +1&  0&  0&  0\cr
nc_2	&  0&  0&  0& -1&  0& +1&  0&  0\cr
}$\vspace*{1cm}\newline
Auf Null kommen:\newline
$y_1=\begin{pmatrix}
	c_1 + c_2 + tok_1 + tok_2\cr
\end{pmatrix}$\newline
$y_2=\begin{pmatrix}
	c_1 + nc_1 + req_1\cr
\end{pmatrix}$\newline
$y_3=\begin{pmatrix}
	c_2 + nc_2 + req_2\cr
\end{pmatrix}$\newline
Es genügt zu zeigen, dass, wenn $M(c_1)=1$ $M(c_2)=0$\newline
darum wähle $y_1$:\newline
Effekt von $e_1$: $+1, 0, -1, 0 = 0$ symmetrisch zu $e_2$\newline
Effekt von $l_1$: $-1, 0, 0, +1 = 0$ symmetrisch zu $l_2$\newline
Effekt von $g_1$: $0, 0, -1, +1 = 0$ symmetrisch zu $g_2$\newline

\subsubsection*{b)}
Das Netz gilt als sicher, wenn gilt: $y^TM_N = 1$\newline
$\Rightarrow \begin{pmatrix}0 & 0 & 1 & 1 & 1 & 1 & 0 & 0\end{pmatrix}\begin{pmatrix}0\cr 0\cr 0\cr 0\cr 1\cr 0\cr 1\cr 1\cr\end{pmatrix}=0+0+0+0+1+0+0+0 = 1$\newline
Selbige Rechnung für $y_2$ und $y_3$ durchführen\newline
$\Rightarrow$ Das Netz ist damit sicher.

\subsubsection*{c)}
Alle $r_1$\newline
Annahme:				\tabto{2.5cm}$v r_1 w$ $e_1 \notin w$\newline
Nach $r_1$:				\tabto{2.5cm}$M(req)=1$ wegen $y_2 \Rightarrow M(c_1)=M(nc_1)=0$\newline
Fälle:	\tabto{1.5cm}i)	\tabto{2.5cm}$req_1 + nc_2 + tok_1$: $e_1$ darf nicht schalten\newline
						\tabto{2.5cm}$r_2 \Rightarrow$ ii) $r_2$ schaltet wegen Maximalität\newline
		\tabto{1.5cm}ii)\tabto{2.5cm}$req_1 + req_2 + tok_1$: nur $e_1$ aktiv\newline
						\tabto{2.5cm}(wegen Annahme Maximailtät müsste schalten $\lightning$)\newline
		\tabto{1.5cm}iii)\tabto{2.5cm}$req_1 + nc_2 + tok_2$: $r_2$ aktiv $\Rightarrow$ iv) $g_2$ aktiv $\Rightarrow$ i)\newline
		\tabto{1.5cm}iv)\tabto{2.5cm}$req_1 + req_2 + tok_2$: $e_2$ aktiv $\Rightarrow$ v)\newline
		\tabto{1.5cm}v)\tabto{2.5cm}$req_1 + c_2$: $l_2$ aktiv $\Rightarrow$ i)\newline
$\Rightarrow$ keine Maximale\newline
stark fair $\Rightarrow$ ist schwach fair $\Rightarrow$ ist maximal

\subsection*{Aufgabe 2}
$M_N(s_0)=0$ nach $t_1$ tot\newline
$M_N(s_0)=1$ $(t_1 t_3 t_2 t_4)^\omega$ lebendig\newline
$M_N(s_0)>1$ $(t_1 t_3 t_2 t_3) \rightarrow tot$\newline

\section{Übung 6}
\subsection*{Aufgabe 1}
Markierungsaufbau: r r1 r2 a1 a2 a\newline
\begin{figure}[H]
\begin{tikzpicture}
	\node at (0,0)		(m1) {0,0,0,0,0,0};
	\node at (0,-2) 	(m2) {1,0,0,0,0,0}
		edge [pre] node [right] {r+} (m1);
	\node at (-2, -4)	(m3) {1,1,0,0,0,0}
		edge [pre] node [left] {r1+} (m2);
	\node at (+2, -4)	(m4) {1,0,1,0,0,0}
		edge [pre] node [right] {r2+} (m2);
	\node at (-2, -6)	(m5) {1,1,0,1,0,0}
		edge [pre] node [left] {a1+} (m3);
	\node at (+2, -6)	(m6) {1,0,1,0,1,0}
		edge [pre] node [right] {a2+} (m4);
	\node at (-2, -8)	(m7) {1,0,0,1,0,0}
		edge [pre] node [left] {r1-} (m5);
	\node at (+2, -8)	(m8) {1,0,0,0,1,0}
		edge [pre] node [right] {r2-} (m6);
	\node at (-2, -10)	(m9) {1,0,0,0,0,0}
		edge [pre] node [left] {a1-} (m7);
	\node at (+2, -10)	(m10) {1,0,0,0,0,0}
		edge [pre] node [right] {a2-} (m8);
	\node at (0, -12)	(m11) {1,0,0,0,0,1}
		edge [pre] node [left] {a+} (m9)
		edge [pre] node [right]  {a+} (m10);
	\node at (0, -14)	(m12) {0,0,0,0,0,1}
		edge [pre] node [right] {r-} (m11)
		edge [post, out=180, in=180] node [left] {a-} (m1);
\end{tikzpicture}
\end{figure}

\subsection*{Aufgabe 2}
\begin{figure}[H]
\begin{tikzpicture}[node distance=1.3cm, >=stealth', bend angle=45, auto]
  \tikzstyle{place}=[circle, thick, draw=blue!75, fill=blue!20, minimum size=6mm]
  \tikzstyle{red place}=[place, draw=red!75, fill=red!20]
  \tikzstyle{transition}=[rectangle, thick, draw=black!75, fill=black!20, minimum size=4mm]
  \tikzstyle{every label}=[red]
  
	\node 	[place, tokens=3]	at	(0,0)		(z)		{};
	\node	[transition]		at	(-4,-2)		(t2)	{$t_2$}
		edge [pre]	(z);
	\node 	[place]				at	(-6,-2)		(s)		{}
		edge [pre]	(t2);
	\node	[transition]		at	(-8,-4)		(t3)	{$t_3$}
		edge [pre]	(s)
		edge [post, bend left]	(z);
	\node 	[place, tokens=2]	at	(-6,-6)		(spass)	{}
		edge [pre]	(t3);
	\node	[transition]		at	(-4,-6)		(t1)	{$t_1$}
		edge [pre]	(spass);
	\node 	[place]				at	(-0,-6)		(wschr)	{}
		edge [pre]	(t1);
	\node	[transition]		at	(2, -4)		(t2')	{$t_2'$}
		edge [pre]	(wschr);
	\node 	[place]				at	(2,-2)		(s1)	{}
		edge [pre]	(t2');
	\node	[transition]		at	(-0, -2)	(t2'')	{$t_2''$}
		edge [pre]	(s1);
	\node 	[place]				at	(-2,-2)		(s2)	{}
		edge [pre]	(t2'')
		edge [post]	(t2);
	\node	[place, tokens=1]	at	(-2,-4)		(s3)	{}
		edge [pre]	(t2)
		edge [post]	(t2'');
\end{tikzpicture}
\end{figure}
S-Invarianten:\newline
wenn bei $s$ eine Marke liegt, liest und schreibt kein anderer Prozess.\newline
$z+l+3s+2s_2+s1=3$\newline
ursprünglich $z+l+3s=3$\newline
wenn auf $s$ mindestens eine Marke liegt $\Rightarrow 3 \Rightarrow$ alle anderen $0$\newline
Beweis:\newline
$\bordermatrix{
		&	t_2'	&	t_2''	&	t_2	&	t_3	&	t_5	&	t_6	\cr
3s		&	0		&	0		&	3	&	-3	&	0	&	0	\cr
2s_2	&	0		&	2		&	-2	&	0	&	0	&	0	\cr
s_1		&	1		&	-1		&	0	&	0	&	0	&	0	\cr
z		&	-1		&	-1		&	-1	&	3	&	-1	&	+1	\cr
l		&	0		&	0		&	0	&	0	&	-1	&	+1\cr
}$\newline
Die Summen unter allen Transitionen ergibt $0$
\newpage
\subsection*{Aufgabe 3}
S-Invarianten:\newline
$y_1=s_0+s_1=n$\newline
$y_2=s_5+s_6=1$\newline
$y_3=s_2+s_3+s_4=1$\newline
Effekte auf\newline
$y1$:	\tabto{1.5cm}$t_3$:\tabto{2cm}$-1+1=0$\newline
		\tabto{1.5cm}$t_4$:\tabto{2cm}$+1-1=0$\newline
$y2$:	\tabto{1.5cm}$t_1$:\tabto{2cm}$+1-1=0$\newline
		\tabto{1.5cm}$t_2$:\tabto{2cm}$-1+1=0$\newline
$y3$:	\tabto{1.5cm}$t_1$:\tabto{2cm}$-1+1+0=0$\newline
		\tabto{1.5cm}$t_2$:\tabto{2cm}$0+1-1=0$\newline
		\tabto{1.5cm}$t_3$:\tabto{2cm}$0-1+1=0$\newline
		\tabto{1.5cm}$t_4$:\tabto{2cm}$+1-1+0=0$\newline

\subsection*{Aufgabe 4}
\subsubsection*{a)}
0-lebendig = tot\newline
4-lebendig $\Rightarrow$ 3-lebendig $\Rightarrow$ 2-lebendig $\Rightarrow$ 1-lebendig\newline
$\neg 3$-lebendig $\Rightarrow$ $\neg 4$-legendig\newline
$\neg 3$-lebendig $\Rightarrow$ Schaltfolgen beinhalten endlich viele t's\newline
$\Rightarrow$ irgendwann kann t nicht mehr schalten $\lightning 4$-lebendig $\Rightarrow$ $\neg 4$-lebendig\newline
3-lebendig $\not\Rightarrow$ 4-lebendig\newline
\begin{tikzpicture}[node distance=1.3cm, >=stealth', bend angle=45, auto]
  \tikzstyle{place}=[circle, thick, draw=blue!75, fill=blue!20, minimum size=6mm]
  \tikzstyle{red place}=[place, draw=red!75, fill=red!20]
  \tikzstyle{transition}=[rectangle, thick, draw=black!75, fill=black!20, minimum size=4mm]
  \tikzstyle{every label}=[red]
	\node[place, tokens=1] 	at(0,0) (s1){};
	\node[transition]		at(0,-2)(t1){t}
		edge[pre, bend left] (s1)
		edge[post, bend right](s1);
	\node[transition]		at(2,0)(t2){v}
		edge[pre] (s1);
\end{tikzpicture} ist zwar 3-lebendig jedoch nicht aus jeder erreichbaren Markierung ist t schaltbar, da v ein Verbraucher ist $\Rightarrow$ Marke weg!\newline
1-lebendig $\not\Rightarrow$ 2-lebendig\newline
\begin{tikzpicture}[node distance=1.3cm, >=stealth', bend angle=45, auto]
  \tikzstyle{place}=[circle, thick, draw=blue!75, fill=blue!20, minimum size=6mm]
  \tikzstyle{red place}=[place, draw=red!75, fill=red!20]
  \tikzstyle{transition}=[rectangle, thick, draw=black!75, fill=black!20, minimum size=4mm]
	\node[place, tokens=1]	at(0,0) (s1) 	{};
	\node[transition]		at(2,0)	(t)		{t}
		edge[pre](s1);
\end{tikzpicture} 1x schalten ja; jedoch $n=2 \lightning$\newline
\newpage
2-lebendig n-mal schalten ja jedoch nicht unendlich oft\newline
\begin{tikzpicture}[node distance=1.3cm, >=stealth', bend angle=45, auto]
  \tikzstyle{place}=[circle, thick, draw=blue!75, fill=blue!20, minimum size=6mm]
  \tikzstyle{red place}=[place, draw=red!75, fill=red!20]
  \tikzstyle{transition}=[rectangle, thick, draw=black!75, fill=black!20, minimum size=4mm]
	\node[place, tokens=1]	at(0,0)			(s1)	{};
	\node[transition]		at(0,-2)		(t1)	{$t_1$}
		edge[pre] (s1);
	\node[place]			at(0,-4)		(s2)	{}
		edge[pre] (t1);
	\node[transition]		at(2,0)			(t2)	{$t_2$}
		edge[pre, bend left]	(s1)
		edge[post, bend right]	(s1);
	\node[place]			at(2,-2)		(s3)	{}
		edge[pre] (t2);
	\node[transition]		at(2,-4)			(t3)	{$t_3$}
		edge[pre] (s3)
		edge[pre, bend left] (s2)
		edge[post, bend right] (s2);
\end{tikzpicture} 42 mal aufbauen; danach 42 mal abbauen,
\subsubsection*{b)}
ist 0-lebendig dann auch $N'$ 0-lebendig? ja\newline
1-lebendig: ja\newline
2-lebendig: ja\newline
3-lebendig: ja\newline
4-lebendig: nein\newline

\newpage

\section{Übung 7}
\subsection*{Aufgabe 1}
\begin{figure}[H]
	\begin{tikzpicture}[node distance=1.3cm, >=stealth', bend angle=45, auto]
	  \tikzstyle{place}=[circle, thick, draw=blue!75, fill=blue!20, minimum size=6mm]
	  \tikzstyle{red place}=[place, draw=red!75, fill=red!20]
	  \tikzstyle{transition}=[rectangle, thick, draw=black!75, fill=black!20, minimum size=4mm]
	  
		\node[place, tokens=1, label={$Pc_6$}]		at(0,0)		(pc6)	{};
		\node[transition]							at(0,-2)	(x4)	{$x_4$}
			edge[post]	(pc6);
		\node[place, tokens=1, label={right:$Pc_1$}]at (0,-4)	(pc1)	{}
			edge[post]	(x4);
		\node[transition]							at(-2,-6)	(x1)	{$x_1$}
			edge[pre]	(pc1);
		\node[transition]							at(0,-6)	(x2)	{$x_2$}
			edge[pre]	(pc1);
		\node[transition]							at(2,-6)	(x3)	{$x_3$}
			edge[pre]	(pc1);
		\node[place, label={right:$Pc_2$}]			at(0,-8)	(pc2)	{}
			edge[pre]	(x1)
			edge[pre]	(x2)
			edge[pre]	(x3);
		\node[transition]							at(-4,-10)	(x11)	{$x_1$}
			edge[pre]	(pc2);
		\node[transition]							at(-2,-10)	(x12)	{$x_2$}
			edge[pre]	(pc2);
		\node[transition]							at(2,-10)	(x13)	{$x_3$}
			edge[pre]	(pc2);
		\node[transition]							at(4,-10)	(x14)	{$x_3$}
			edge[pre]	(pc2);
		\node[place, label={left:$Pc_3$}]			at(-3, -12)	(pc3)	{}
			edge[pre]	(x11)
			edge[pre]	(x12);
		\node[place, label={left:$Pc_4$}]			at(3, -12)	(pc4)	{}
			edge[pre]	(x13)
			edge[pre]	(x14);
		\node[transition]							at(-5, -14)	(t1)	{}
			edge[pre]	(pc3);
		\node[transition]							at(-3, -14)	(t2)	{}
			edge[pre]	(pc3);
		\node[transition]							at(-1, -14)	(t3)	{}
			edge[pre]	(pc3);
		\node[transition]							at(1, -14)	(t4)	{}
			edge[pre]	(pc4);
		\node[transition]							at(3, -14)	(t5)	{}
			edge[pre]	(pc4);
		\node[transition]							at(5, -14)	(t6)	{}
			edge[pre]	(pc4);
		\node[transition]							at(7, -14)	(t7)	{}
			edge[pre]	(pc4);
		\node[place, label={right:$x_1$}]			at(-4,-16)	(sx1)	{}
			edge[pre]	(t1)
			edge[post]	(t4);
		\node[place, label={right:$x_2$}]			at(-2,-16)	(sx2)	{}
			edge[pre]	(t1)
			edge[post]	(t5);
		\node[place, label={right:$x_3$}]			at(2,-16)	(sx3)	{}
			edge[pre]	(t2)
			edge[post]	(t6);
		\node[place, label={right:$x_4$}]			at(4,-16)	(sx4)	{}
			edge[pre]	(t3)
			edge[pre, out=60, in=-180]	(t7)
			edge[post, out=0, in =-100]	(t7);
		\node[place, label={$Pc_5$}]				at(0,-18)	(pc5)	{}
			edge[pre]	(t1)
			edge[pre]	(t2)
			edge[pre]	(t3)
			edge[pre]	(t4)
			edge[pre]	(t5)
			edge[pre]	(t6)
			edge[pre]	(t7);
	\end{tikzpicture}
\end{figure}OH MANN!!!

\subsection*{Aufgabe 2}
\subsubsection*{a)}
Satz 4.5 Transitionen können sehen ob eine andere geschalten hat wenn dazwischen eine Stelle ist die eine Marke beinhaltet.\newline
\end{document}