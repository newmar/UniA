\documentclass[12pt]{scrreprt}
\usepackage{acronym}
\usepackage{amsmath}
\usepackage{amssymb}

\usepackage[ngerman]{babel}

\usepackage{hyperref}

\usepackage{breakurl}

\usepackage{cite}
\usepackage{color}

\usepackage{float}

\usepackage{graphicx}



\usepackage[utf8]{inputenc}

\usepackage{listings}
\usepackage{longtable}

\usepackage{url}

\usepackage{rotating}

\usepackage{setspace}
\usepackage{stmaryrd}
\usepackage{subfigure}

\usepackage{tabularx}
\usepackage{tabto}
\usepackage{threeparttable}
\usepackage{tikz}

\usepackage{units}

\usetikzlibrary{arrows, shapes, decorations, automata, backgrounds, petri}
\urlstyle{same}
\definecolor{dkgreen}{rgb}{0,0.6,0}
\definecolor{gray}{rgb}{0.5,0.5,0.5}
\definecolor{mauve}{rgb}{0.58,0,0.82}
\def\TReg{\textsuperscript{\textregistered}}
\def\TCop{\textsuperscript{\textcopyright}}
\def\TTra{\textsuperscript{\texttrademark}}

\def\colCOL#1{\multicolumn{1}{>{\columncolor{green!30}}c}{#1}}
\def\colCell#1#2{\multicolumn{1}{>{\columncolor{#1}}c}{#2}} 


\title{Zusammenfassung Petrinetze}
\author{B. Sc. Markus Neuerburg}
\date{\today}

\begin{document}
\setlength{\topmargin}{0cm}
\parindent 0pt
\onehalfspacing

\tableofcontents
\newpage
\chapter*{Abkürzungsverzeichnis}
\begin{acronym}[HBCI]
	\acro{aa}[$\mathrm{T^*}$]{akzeptiertes Alphabet}
	\acro{es}[$\mathrm{\lambda}$]{leere Schaltfolge}
\end{acronym}

\chapter{Grundbegriffe}
\begin{itemize}
	\item $t \in T:$						\tabto{4cm} Transition aus der Menge aller Transitionen
	\item $s \in S:$						\tabto{4cm} Stelle aus der Menge aller Stellen
	\item $(x, y) \in F:$					\tabto{4cm} Kannte aus der Menge aller Flussrelationen (Kanten)
	\item $W:$								\tabto{4cm} Gewichtungs-Funktion
	\item $W(x, y):$						\tabto{4cm} Kanntengewicht (Gewicht auf den Pfeilen)
	\item $^\bullet x=\{y \mid (y,x) \in F \}$	\tabto{4cm} Der Vorbereich von x\newline
	Sprich: Vorbereich von x ist y mit der Eigenschaft: Kannte von y nach x ist Element aller Flussrelationen (Pfeile)
	\item $x^\bullet =\{y \mid (x, y) \in F\}$	\tabto{4cm} Der NAchbereich von x\newline
	Sprich: Nachbereich von x ist y mit der Eigenschaft: Kannte von x nach y ist Element aller Flussrelationen (Pfeile)
	\item $M:S\mapsto \mathbb{N}$			\tabto{4cm} Markierung\newline
	Eine Markierung M ist eine Menge von Stellen abgebildet auf $\mathbb{N}$
\end{itemize}

\chapter{Definitionen}
\section{Definition 2.2 (aktivierte Transition)}
$t \in T$ ist aktiviert unter Markierung $M$, $M\left[t\right>$, falls $\forall s \in S:W(s,t) \leq M(s)$\newline
	Sprich: Transition $t$ ist aktiviert unter Markierung $M$, falls für alle Stellen aus der Menge $S$ gilt, dass das Kanntengewicht der Kannte von s nach t kleiner oder gleich Anzahl der Marken auf Stelle $s$ ($M(s)$) ist.
\section{Definition 2.3 (Schaltfolge)}
Sei $w \in$ \ac{aa} : $M\left[w\right>$ bzw. $M\left[w\right>M'$ falls:
\begin{itemize}
	\item $w =$ \ac{es} (und $M=M'$)
	\item $w = w't$ mit $t \in T$, $M\left[w'\right>M''\left[t'\right>$ (und $M''\left[t\right>M'$)
\end{itemize}
$FS(N)=\{w \in$ \ac{aa}$ \mid M_N \left[w\right>\}$ Menge der Schaltfolgen von N (firing sequence)\newline
$w \in T^\omega$ unendliche Schaltfolge (falls alle endlichen Präfixe von w Schaltfolgen sind)

\section{Definition 2.21}
\subsection{tot}
$t$ heißt tot unter $M$, falls $\forall M' \in \left[M\right> : \neg M'\left[t\right>$\newline
$t$ heißt tot, falls $t$ tot unter $M_N$.\newline
$M$ heißt tot, falls alle Transitionen tot unter $M$ sind.
\newpage
\subsection{lebendig}
$t$ heißt lebendig unter $M$, falls $t$ unter keiner von $M$ erreichbaren Markierung $M'$ tot ist:\newline
$\forall M' \in \left[M\right>\exists M'' \in \left[M'\right> :  M'' \left[t\right>$\newline
$M$ heißt lebendig, wenn alle $t$ unter $M$ lebendig sind.\newline
$t$ heißt lebendig, falls $t$ lebendig unter $M_N$.\newline
$N$ heißt lebendig, falls alle Transitionen lebendig sind.

\subsection{Identifikation von S-Invarianten}
S-Invarianten Sind nützlich um Situationen zu untersuchen bei denen es zu Konflikten kommen kann. z.B. reader writer Prozess. Hier ist zu beachten, dass lesen und schreiben nicht gleichzeitig geschehen dürfen. Dass dies sichergestellt werden kann dürfen zu keinem Zeitpunkt die Marken auf der Stelle für die Zugriffsrechte ausrechen, dass nach dem Starten eines Schreibprozess sofort der Leseprozess geschalten werden kann. Das kann nur der Fall sein, wenn jede Marke auch wirklich verbraucht wird und immer nur soviele Marken auf die Stellen gelegt werden wie auch verbraucht werden. Gleiches gilt auch für den Leseprozess. Hier dürfen jedoch mehrere Leseprozesse hintereinander gestartet werden. Auch hier gilt: Alle Leseprozesse müssen erst beendet werden, dass ein Schreibprozess gestartet werden darf.

\chapter{Übungen}

\section{Übung 1}
\subsection*{Aufgabe 1}
\begin{figure}[H]
\begin{tikzpicture}[node distance=1.3cm, >=stealth', bend angle=45, auto]
  \tikzstyle{place}=[circle, thick, draw=blue!75, fill=blue!20, minimum size=6mm]
  \tikzstyle{red place}=[place, draw=red!75, fill=red!20]
  \tikzstyle{transition}=[rectangle, thick, draw=black!75, fill=black!20, minimum size=4mm]
  \tikzstyle{every label}=[red]

	\begin{scope}
		\node[place, tokens=1] 	at(0,0)		(s1){};
		\node[place]			at(0,-2)	(s2){};
		\node[place]			at(2.1,0)	(s3){};
		\node[place, tokens=1]	at(2.1,-2)	(s4){};
		\node[place, tokens=1]	at(4.1,0)	(s5){};
		\node[place]			at(4.1,-2)	(s6){};
		\node[place]			at(6.1,0)	(s7){};
		\node[place, tokens=1]	at(6.1,-2)	(s8){};
		
		\node[transition]	at(-1,-1)	(t1){}
			edge[pre, bend left]			(s1)
			edge[post,bend right]			(s2);
		\node[transition]	at(1.1,-1)	(t2){}
			edge[pre, bend left]			(s2)
			edge[post,bend right]			(s1)
			edge[pre, bend right]			(s4)
			edge[post,bend left]			(s3);
		\node[transition]	at(3.1,-1)	(t3){}
			edge[pre, bend right]			(s3)
			edge[post,bend left]			(s4)
			edge[pre, bend left]			(s5)
			edge[post,bend right]			(s6);
		\node[transition]	at(5.1,-1)	(t4){}
			edge[pre, bend left]			(s6)
			edge[post,bend right]			(s5)
			edge[pre, bend right]			(s8)
			edge[post,bend left]			(s7);
		\node[transition]	at(7.1,-1)	(t5){}
			edge[pre, bend right]			(s7)
			edge[post,bend left]			(s8);
	\end{scope}
\end{tikzpicture}
\caption{Lösung 1}
\end{figure}
\begin{figure}[H]
\begin{tikzpicture}[node distance=1.3cm, >=stealth', bend angle=45, auto]
  \tikzstyle{place}=[circle, thick, draw=blue!75, fill=blue!20, minimum size=6mm]
  \tikzstyle{red place}=[place, draw=red!75, fill=red!20]
  \tikzstyle{transition}=[rectangle, thick, draw=black!75, fill=black!20, minimum size=4mm]
  \tikzstyle{every label}=[red]

	\begin{scope}
		\node[place, tokens=1] 	at(0,0)		(s1){};
		\node[place]			at(0,-2)	(s2){};
		\node[place]			at(2.1,0)	(s3){};
		\node[place, tokens=2]	at(2.1,-2)	(s4){};
		\node[place, tokens=1]	at(4.1,0)	(s5){};
		\node[place]			at(4.1,-2)	(s6){};
		
		\node[transition]	at(-1,-1)	(t1){}
			edge[pre, bend left]			(s1)
			edge[post,bend right]			(s2);
		\node[transition]	at(1.1,-1)	(t2){}
			edge[pre, bend left]			(s2)
			edge[post,bend right]			(s1)
			edge[pre, bend right]			(s4)
			edge[post,bend left]			(s3);
		\node[transition]	at(3.1,-1)	(t3){}
			edge[pre, bend right]			(s3)
			edge[post,bend left]			(s4)
			edge[pre, bend left]			(s5)
			edge[post,bend right]			(s6);
		\node[transition]	at(5.1,-1)	(t4){}
			edge[pre, bend left]			(s6)
			edge[post,bend right]			(s5);
	\end{scope}
\end{tikzpicture}
\caption{Lösung 2}
\end{figure}
\subsection*{Aufgabe 2}
\subsubsection*{a)}
$n=0:$ $M(1,0,0) \in M_N$\newline
$n=1:$ $M(1,0,1)$\newline
Erreichbar mit:\newline
$M_N\left[a\right>M'\left[b\right>M(1,0,1)$ = $M_N\left[(ab)\right>M(1,0,1)$ $w=(a,b)$\newline
$n=n:$ $M_N\left[(ab)^n\right>M(1,0,n)$
\newpage
\subsubsection*{b)}
%TODO Bild
Erreichbare Markierungen sind:\newline
$(0,0,0) , (1,0,0) , (1,0,1) , (1,0,n) , (1,0,(n+1))$\newline
$(1,0,0)\left[(ab)\right>(1,0,1)$\newline
$(1,0,0)\left[(ab)^n\right>(1,0,n)$\newline
$(1,0,0)\left[(ab)^{n+1}\right>(1,0,(n+1))$\newline
$(1,0,0)\left[(ab)\right>(1,0,1)\left[(c\right>(0,0,(n-1))$\newline
$(1,0,0)\left[(ab)^nc\right>(0,0,n)$\newline

\subsection*{d}
$R={(1,0,n), (0,1,(n+1)), (0,0,n)\mid n \in \mathbb{N}} \subseteq \left[M_N\right>$\newline
$R \supseteq \left[M_N\right>$\newline
Behauptung: $M_N\left[w\right>M\Rightarrow M \in R$ Induktion über w\newline
\newline
Beweis:
\begin{itemize}
	\item $w=\lambda: M=M_N \in R$\textcolor{green}{\checkmark}
	\item $w=w't: M_N\left[w'\right>M'\left[t\right>M$ Nach Induktion $M'\in R$\textcolor{green}{\checkmark}
	\item $M'=(1,0,n): t=a:M={(0,1,(n+1))} \in R$ \textcolor{green}{\checkmark}
\end{itemize}
$t=c \wedge n>=1: M={(0,0,(n-1))} \in R$ \textcolor{green}{\checkmark}\newline
$M'={(0,1,(n+1))}: t=b: M={(1,0,(n+1))} \in R$\textcolor{green}{\checkmark}\newline
$M'={(0,0,n)}: \neg \exists t$  \textcolor{red}{$\lightning$}

\subsection*{Aufgabe 3}
Alle Markierungen:
\begin{figure}[H]
\begin{tikzpicture}
	\node at (0,0) 			(m1){1,1,0,0,0};
	\node at (0,-1.5)		(m2){0,0,1,0,0}
		edge [pre] node[right] {s+} (m1);
	\node at (0,-3)			(m3){0,0,0,1,1}
		edge [pre] node[right] {e+} (m2);
	\node at (-2, -4.5)		(m4){1,0,0,1,0}
		edge [pre] node[left] {s-}(m3)
		edge [post, bend left] node[left] {e-} (m1);
	\node at (2, -4.5)		(m5){0,1,0,0,1}
		edge [pre] node[right] {e-} (m3)
		edge [post, bend right] node[right] {s-} (m1);

\end{tikzpicture}
\end{figure}
Alle Codes:
\begin{figure}[H]
\begin{tikzpicture}
	\node at (0,0) 			(m1){0,0};
	\node at (0,-1.5)		(m2){0,1}
		edge [pre] node[right] {s+} (m1);
	\node at (0,-3)			(m3){1,1}
		edge [pre] node[right] {e+} (m2);
	\node at (-2, -4.5)		(m4){0,1}
		edge [pre] node[left] {s-}(m3)
		edge [post, bend left] node[left] {e-} (m1)
		edge [post, color=red, bend left] (m2);
	\node at (2, -4.5)		(m5){1,0}
		edge [pre] node[right] {e-} (m3)
		edge [post, bend right] node[right] {s-} (m1);
\end{tikzpicture}
\end{figure}
Es besteht ein Konflikt. Zu vermeiden wäre der Konflikt mit einer zusätzlichen Stelle im Code.\newline
Der STG ist jedoch konsistent, da nie s+, e+, s- oder e- zweimal hintereinander oder s+ und s- bzw e+ und e- sofort hintereinander ausgeführt werden.
\newpage

\section{Übung 2}
\subsection*{Aufgabe 2}
\subsubsection*{a)}
Behauptung:\newline
$M_{N'} = M_N + \delta M$ 	\tabto{5cm} $\Rightarrow FS(N) <= FS(N')$\newline
Beweis:\newline
$M_N \left[w\right>M'$ 	\tabto{5cm} $\Leftarrow w \in FS(N)$\newline
$M_N + \delta M \left[w\right>M_N' + \delta M$ \tabto{5cm} $\Rightarrow w \in FS(N')$\newline
Nach 2.7.\newline
$M_{N'} \left[w\right>M_N' + \delta M$

\subsubsection*{b)}
$s'=S-\left\lbrace s_0\right\rbrace, W' = W \mid _{S-\left\lbrace s_0 \right\rbrace}, M_{N'} = M_N \mid  _{S-\left\lbrace s_0\right\rbrace}$\newline
$M_N\left[w\right>M \Rightarrow M_{N'}\left[w\right>M' \wedge M' = M \mid _S-\left\lbrace s_0 \right\rbrace$\newline
Induktionsanfang:\newline
$w=\lambda:$\tabto{2cm}$M_{N} \left[w\right> M \Rightarrow M_N = M$\newline
			\tabto{2cm}$M_{N'} \left[w\right> M' \Rightarrow M_{N'} = M \mid _S-\left\lbrace s_0 \right\rbrace$\newline
Induktionsschritt:\newline
$w=w't:$\tabto{2cm}$M_N \left[w'\right>M''$\newline
		\tabto{2cm}$M_{N' \left[w'\right>}M''' \wedge M''' = M'' \mid _S-\left\lbrace s_0 \right\rbrace$ nach Induktion\newline
Die Markierungen sind die Selben. Das 2.Netz hat lediglich eine Stelle weniger\newline
Zu zeigen: \tabto{2cm} $M''\left[t\right>M \Rightarrow M'''\left[t\right>M' \wedge M' = M \mid _S-\left\lbrace s_0 \right\rbrace$

\newpage

\section{Übung 3}
\subsection*{Aufgabe 1}
Schalten von 0 Transitionen:\newline
$M_N \left[\lambda\right> M$ \tabto{4cm} $M_N = M = {1,2,1}$\newline
Schalten von 1 Transition:\newline
Schalten von a b oder c:\newline
${(1,2,1)} \left[a\right> {(0,1,1)}$\newline
${(1,2,1)} \left[b\right> {(1,1,0)}$\newline
${(1,2,1)} \left[c\right> {(1,2,0)}$\newline
${(1,2,1)} \left[d\right> {(1,1,2)}$\newline
Da es sich um eine Halbordnung handelt muss jedes Element einer Markierung mit den anderen Elementen der anderen Markierungen verglichen werden. Daraus resultiert, dass es sich bei $(1,1,2)$ und $(1,2,1)$ um maximale Markierungen handelt.

\section{Übung 4}
\subsection*{Aufgabe 1}
$N_1:$	\tabto{2cm} ist lebendig\newline
		\tabto{2cm} ist verklemmungsfrei\newline
$N_2:$	\tabto{2cm} ist nicht tot\newline
		\tabto{2cm} ist nicht verklemmungsfrei\newline
$N_3:$	\tabto{2cm} ist lebendig\newline
		\tabto{2cm} ist nicht verklemmungsfrei\newline
$N_4:$	\tabto{2cm} ist lebendig\newline
		\tabto{2cm} ist nicht verklemmungsfrei\newline
$N_5:$	\tabto{2cm} ist tot\newline
		\tabto{2cm} ist nicht verklemmungsfrei\newline

\subsection*{Aufgabe 2}
Es gibt keine lebendige Markierung

\subsection*{Aufgabe 4}
$r_1(r_2 e_2 L_2)^\omega$\newline
schwach fair, da $e_1$ nicht fair behandelt wird\newline
$e_2$ ist zwar unendlich oft aktiviert aber auch unendlich oft deaktiviert.\newline
stark fair:\newline
$r_1$ und $l_1$ sind nie aktiviert ($\Rightarrow$ können sich nicht beschweren, tun nichts zur Sache)\newline
S-Invariante $\Rightarrow$ $r_1$ leer $l_1$ immer leer\newline
$r_2 e_2 l_2$: $e_1$ wird nicht stark fair behandelt.,

\newpage

\section{Übung 5}
\subsection*{Aufgabe 1}
\subsubsection*{a)}
Suchen der S-Invarianten:\newline
Inzidenzmatrix:\newline
$\bordermatrix{ %
		&e_1&e_2&r_1&r_2&l_1&l_2&g_1&g_2\cr
req_1	& -1&  0& +1&  0&  0&  0&  0&  0\cr
req_2	&  0& -1&  0& +1&  0&  0&  0&  0\cr
c_1		& +1&  0&  0&  0& -1&  0&  0&  0\cr
c_2		&  0& +1&  0&  0&  0& -1&  0&  0\cr
tok_1	& -1&  0&  0&  0&  0& +1& -1& +1\cr
tok_2	&  0& -1&  0&  0& +1&  0& +1& -1\cr
nc_1	&  0&  0& -1&  0& +1&  0&  0&  0\cr
nc_2	&  0&  0&  0& -1&  0& +1&  0&  0\cr
}$\vspace*{1cm}\newline
Auf Null kommen:\newline
$y_1=\begin{pmatrix}
	c_1 + c_2 + tok_1 + tok_2\cr
\end{pmatrix}$\newline
$y_2=\begin{pmatrix}
	c_1 + nc_1 + req_1\cr
\end{pmatrix}$\newline
$y_3=\begin{pmatrix}
	c_2 + nc_2 + req_2\cr
\end{pmatrix}$\newline
Es genügt zu zeigen, dass, wenn $M(c_1)=1$ $M(c_2)=0$\newline
darum wähle $y_1$:\newline
Effekt von $e_1$: $+1, 0, -1, 0 = 0$ symmetrisch zu $e_2$\newline
Effekt von $l_1$: $-1, 0, 0, +1 = 0$ symmetrisch zu $l_2$\newline
Effekt von $g_1$: $0, 0, -1, +1 = 0$ symmetrisch zu $g_2$\newline

\subsubsection*{b)}
Das Netz gilt als sicher, wenn gilt: $y^TM_N = 1$\newline
$\Rightarrow \begin{pmatrix}0 & 0 & 1 & 1 & 1 & 1 & 0 & 0\end{pmatrix}\begin{pmatrix}0\cr 0\cr 0\cr 0\cr 1\cr 0\cr 1\cr 1\cr\end{pmatrix}=0+0+0+0+1+0+0+0 = 1$\newline
Selbige Rechnung für $y_2$ und $y_3$ durchführen\newline
$\Rightarrow$ Das Netz ist damit sicher.

\subsubsection*{c)}
Alle $r_1$\newline
Annahme:				\tabto{2.5cm}$v r_1 w$ $e_1 \notin w$\newline
Nach $r_1$:				\tabto{2.5cm}$M(req)=1$ wegen $y_2 \Rightarrow M(c_1)=M(nc_1)=0$\newline
Fälle:	\tabto{1.5cm}i)	\tabto{2.5cm}$req_1 + nc_2 + tok_1$: $e_1$ darf nicht schalten\newline
						\tabto{2.5cm}$r_2 \Rightarrow$ ii) $r_2$ schaltet wegen Maximalität\newline
		\tabto{1.5cm}ii)\tabto{2.5cm}$req_1 + req_2 + tok_1$: nur $e_1$ aktiv\newline
						\tabto{2.5cm}(wegen Annahme Maximailtät müsste schalten $\lightning$)\newline
		\tabto{1.5cm}iii)\tabto{2.5cm}$req_1 + nc_2 + tok_2$: $r_2$ aktiv $\Rightarrow$ iv) $g_2$ aktiv $\Rightarrow$ i)\newline
		\tabto{1.5cm}iv)\tabto{2.5cm}$req_1 + req_2 + tok_2$: $e_2$ aktiv $\Rightarrow$ v)\newline
		\tabto{1.5cm}v)\tabto{2.5cm}$req_1 + c_2$: $l_2$ aktiv $\Rightarrow$ i)\newline
$\Rightarrow$ keine Maximale\newline
stark fair $\Rightarrow$ ist schwach fair $\Rightarrow$ ist maximal

\subsection*{Aufgabe 2}
$M_N(s_0)=0$ nach $t_1$ tot\newline
$M_N(s_0)=1$ $(t_1 t_3 t_2 t_4)^\omega$ lebendig\newline
$M_N(s_0)>1$ $(t_1 t_3 t_2 t_3) \rightarrow tot$\newline


\end{document}