\documentclass[12pt]{scrreprt}
\usepackage{acronym}
\usepackage{amsmath}
\usepackage{amssymb}

\usepackage[ngerman]{babel}
\usepackage{breakurl}

\usepackage{cite}
\usepackage{color}

\usepackage{float}

\usepackage{graphicx}

\usepackage{hyperref}

\usepackage[utf8]{inputenc}

\usepackage{listings}
\usepackage{longtable}

\usepackage{url}

\usepackage{rotating}

\usepackage{setspace}
\usepackage{subfigure}

\usepackage{tabularx}
\usepackage{tabto}
\usepackage{threeparttable}
\usepackage{tikz}

\usepackage{units}

\urlstyle{same}
\definecolor{dkgreen}{rgb}{0,0.6,0}
\definecolor{gray}{rgb}{0.5,0.5,0.5}
\definecolor{mauve}{rgb}{0.58,0,0.82}
\def\TReg{\textsuperscript{\textregistered}}
\def\TCop{\textsuperscript{\textcopyright}}
\def\TTra{\textsuperscript{\texttrademark}}


\title{Zusammenfassung Petrinetze}
\author{B. Sc. Markus Neuerburg}
\date{\today}

\begin{document}
\setlength{\topmargin}{0cm}
\parindent 0pt
\onehalfspacing

\tableofcontents
\newpage
\chapter*{Abkürzungsverzeichnis}
\begin{acronym}[HBCI]
	\acro{aa}[$\mathrm{T^*}$]{akzeptiertes Alphabet}
	\acro{es}[$\mathrm{\lambda}$]{leere Schaltfolge}
\end{acronym}

\chapter{Grundbegriffe}
\begin{itemize}
	\item $t \in T:$						\tabto{4cm} Transition aus der Menge aller Transitionen
	\item $s \in S:$						\tabto{4cm} Stelle aus der Menge aller Stellen
	\item $(x, y) \in F:$					\tabto{4cm} Kannte aus der Menge aller Flussrelationen (Kanten)
	\item $W:$								\tabto{4cm} Gewichtungs-Funktion
	\item $W(x, y):$						\tabto{4cm} Kanntengewicht (Gewicht auf den Pfeilen)
	\item $^\bullet x=\{y \mid (y,x) \in F \}$	\tabto{4cm} Der Vorbereich von x\newline
	Sprich: Vorbereich von x ist y mit der Eigenschaft: Kannte von y nach x ist Element aller Flussrelationen (Pfeile)
	\item $x^\bullet =\{y \mid (x, y) \in F\}$	\tabto{4cm} Der NAchbereich von x\newline
	Sprich: Nachbereich von x ist y mit der Eigenschaft: Kannte von x nach y ist Element aller Flussrelationen (Pfeile)
	\item $M:S\mapsto \mathbb{N}$			\tabto{4cm} Markierung\newline
	Eine Markierung M ist eine Menge von Stellen abgebildet auf $\mathbb{N}$
	\item 
	\item 
\end{itemize}

\chapter{Definitionen}
\section{Definition 2.2 (aktivierte Transition)}
$t \in T$ ist aktiviert unter Markierung $M$, $M\left[t\right>$, falls $\forall s \in S:W(s,t) \leq M(s)$\newline
	Sprich: Transition $t$ ist aktiviert unter Markierung $M$, falls für alle Stellen aus der Menge $S$ gilt, dass das Kanntengewicht der Kannte von s nach t kleiner oder gleich Anzahl der Marken auf Stelle $s$ ($M(s)$) ist.
\section{Definition 2.3 (Schaltfolge)}
Sei $w \in$ \ac{aa} : $M\left[w\right>$ bzw. $M\left[w\right>M'$ falls:
\begin{itemize}
	\item $w =$ \ac{es} (und $M=M'$)
	\item $w = w't$ mit $t \in T$, $M\left[w'\right>M''\left[t'\right>$ (und $M''\left[t\right>M'$)
\end{itemize}
$FS(N)=\{w \in$ \ac{aa}$ \mid M_N \left[w\right>\}$ Menge der Schaltfolgen von N (firing sequence)\newline
$w \in T^\omega$ unendliche Schaltfolge (falls alle endlichen Präfixe von w Schaltfolgen sind)
\chapter{Übungen}

\section{Übung 1}

\end{document}