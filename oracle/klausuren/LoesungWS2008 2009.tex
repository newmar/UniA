\documentclass[12pt]{scrreprt}
\usepackage{acronym}
\usepackage{amsmath}
\usepackage{float}
\usepackage{amssymb}
\usepackage[utf8]{inputenc}
\usepackage[ngerman]{babel}
\usepackage{url}
\usepackage{tabularx}
\usepackage{longtable}
\usepackage{graphicx}
\usepackage{subfigure}
\usepackage{units}
\usepackage{setspace}
\usepackage{tikz}
\title{Lawinensimulation}
\author{B.Sc. Markus Neuerburg}
\date{16.05.2012}

\begin{document}
\section*{Aufgabe 1: Oracle-SQL, 9 Punkte (3+2+4)}
\subsection*{a)}
SELECT datum GREATEST(wert1, wert2, wert3, wert4, wert5, wert6) gr,\newline
\hspace*{5mm}LEAST(wert1, wert2, wert3, wert4, wert5, wert6) l\newline
\hspace*{5mm}FROM messwerte;\newline
\subsection*{b)}
SELECT datum, wert1, wert1-LAG(wert1, 1)\newline
\hspace*{5mm}(ORDER BY datum);\newline
\subsection*{c)}
CREATE ROLE R1\newline
CREATE USER leser NOT IDENTIFIED\newline
GRANT select(datum, wert) ON R1 TO leser\newline
\section*{Aufgabe 2: Trigger, 8 Punkte(2+6)}
\subsection*{a)}
CREATE TRIGGER OR REPLACE triger1\newline
\hspace*{5mm}BEFORE UPDATE ON angestellte\newline
\hspace*{5mm}WHEN new.gehalt != old.gehalt\newline
Begin\newline
\hspace*{5mm}INSERT INTO archive(:old.pers\_id, :old.gehaklt, :new.gehalt, to\_date(now()), :new.kommentar)\newline
END\newline
\subsection*{b)}
CREATE TRIGGER OR REPLACE trigger2\newline
\hspace*{5mm}AFTER INSERT ON archive\newline
\hspace*{5mm}WHEN old.gehalt=new.gehalt\newline
Begin\newline
\hspace*{5mm}UPDATE archive SET kommentar = 'Gehalt identisch zum' $||$ :old.datum\newline
\hspace*{5mm}WHERE :old.pers\_id = :new.pers\_id AND\newline
\hspace*{5mm}datum = :new.datum\newline
\section*{Aufgabe 3 Objekte in Oracle, 9 Punkte (3+2+2+2)}
\subsection*{a)}
CREATE OR REPLACE type\_zimmer\newline
CREATE TYPE zimmer\_t AS OBJECT(nummer INTEGER, bettenanzahl INTEGER);\newline
CREATE TYPE zimmer\_tab IS TABLE OF zimmer\_type;\newline
CREATE TABLE hotel(tel INTEGER, name VARCHAR(20), zimmer zimmer\_tab);\newline
NASTED TABLE zimmer STORE AS hotel\_zimmer;\newline
\subsection*{b)}
INSERT INTO hotel VALUES(\newline
\hspace*{5mm}5550101, 'Zum Orakel',\newline
\hspace*{5mm}zimmer\_tab(1,2),\newline
\hspace*{5mm}zimmer\_tab(1,2),\newline
\hspace*{5mm}zimmer\_tab(1,2));\newline
\subsection*{c)}
SELECT zimmer\_tab.nummer \newline
\hspace*{5mm}WHERE zimmer\_tab.bettenanzahl=2\newline
\hspace*{5mm}AND name='Zum Orakel'\newline
\hspace*{5mm}FROM hotel;\newline
\subsection*{d)}
SELECT name, SUM(zimmer\_tab.bettenanzahl) s FROM hotel h,\newline
\hspace*{5mm}TABLE(h.zimmer) z ORDER BY s desc;
\section*{Aufgabe 4: Rekursion in Orakle, 6 Punkte (3+1+2)}
\subsection*{a)}
SELEKT parent, child FROM tree\newline
\hspace*{5mm}START WITH parent = 1\newline
\hspace*{5mm}CONNECT BY parent = PRIOR child;\newline
\subsection*{b)}
SELECT child, LEVEL FROM tree\newline
\hspace*{5mm}CONNECT BY parent = PRIOR child;\newline
\subsection*{c)}
SELECT child, CONNECT\_BY\_ISLEAF\newline
\hspace*{5mm}FROM tree;
\end{document}