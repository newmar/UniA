\documentclass[12pt]{scrreprt}
\usepackage{hyperref}
\usepackage{acronym}
\usepackage{amsmath}
\usepackage{float}
\usepackage{amssymb}
\usepackage[utf8]{inputenc}
\usepackage[ngerman]{babel}
\usepackage{url}
\usepackage{breakurl}
\usepackage{tabularx}
\usepackage{longtable}
\usepackage{threeparttable}
\usepackage{setspace}
\usepackage{graphicx}
\usepackage{rotating}
\usepackage{subfigure}
\usepackage{units}
\usepackage{setspace}
\usepackage{cite}
\usepackage{color}
\usepackage{listings}
\usepackage{tikz}
\urlstyle{same}
\definecolor{dkgreen}{rgb}{0,0.6,0}
\definecolor{gray}{rgb}{0.5,0.5,0.5}
\definecolor{mauve}{rgb}{0.58,0,0.82}
\def\TReg{\textsuperscript{\textregistered}}
\def\TCop{\textsuperscript{\textcopyright}}
\def\TTra{\textsuperscript{\texttrademark}}

\title{Zusammenfassung Oracle}
\author{B.Sc. Markus Neuerburg}
\date{31.01.2013}

\lstset{ %
  language=SQL,                % the language of the code
  basicstyle=\footnotesize,           % the size of the fonts that are used for the code
  numbers=left,                   % where to put the line-numbers
  numberstyle=\tiny\color{gray},  % the style that is used for the line-numbers
  stepnumber=1,                   % the step between two line-numbers. If it's 1, each line 
                                  % will be numbered
  numbersep=5pt,                  % how far the line-numbers are from the code
  boxpos=c,
  backgroundcolor=\color{white},      % choose the background color. You must add \usepackage{color}
  showspaces=false,               % show spaces adding particular underscores
  showstringspaces=false,         % underline spaces within strings
  showtabs=false,                 % show tabs within strings adding particular underscores
  frame=single,                   % adds a frame around the code
  rulecolor=\color{black},        % if not set, the frame-color may be changed on line-breaks within not-black text (e.g. commens (green here))
  tabsize=2,                      % sets default tabsize to 2 spaces
  captionpos=b,                   % sets the caption-position to bottom
  breaklines=true,                % sets automatic line breaking
  breakatwhitespace=false,        % sets if automatic breaks should only happen at whitespace
  title=\lstname,                   % show the filename of files included with \lstinputlisting;
                                  % also try caption instead of title
  keywordstyle=\color{blue},          % keyword style
  commentstyle=\color{dkgreen},       % comment style
  stringstyle=\color{mauve},         % string literal style
  escapeinside={\%*}{*)},            % if you want to add LaTeX within your code
  morekeywords={*,...}               % if you want to add more keywords to the set
}


\begin{document}
\setlength{\topmargin}{0cm}
\parindent 0pt
\chapter{Zugriffsrechte}
\section{SQL-Injection}
\lstset{caption={Code gegen SQL-Injection},label=DescriptiveLabel}
\begin{lstlisting}
	String sql = "UPDATE FEEDBACK SET INTERACTION_END=" + d.getTime()
	+ ", MINEOPINION=" ...
	...
	+ sanitizeString(myChoiceReason.getValue())
	...
\end{lstlisting}
UPDATE FEEDBACK SET\newline
INTERACTION\_END="22102012",\newline
MINEOPINION="5",...
MYEXPLANATION='mir gefällt dieses Buch',\newline
\hspace*{5mm}\textcolor{red}{got you where 1=1;}\newline
where participant='Markus';\newline
\newline
Um die Gefahr einer Sql-Injection zu minimieren $\Rightarrow$ Berechtigungen beschränken\newline
\newline
\section{Benutzerrechte}
Identifikation: durch Benutzername\newline
Attribute:
\begin{itemize}
	\item Authentifizierungsmethode (intern, extern)\hfill \\
	extern: in einem externen System z.B. Passwortserver
	\item Passwort zur Authentifizierung (verschlüsselt gespeichert)
	\item Voreingestellte Tabelspaces für temporären und dauerhaften Speicher
	\item Tabelspace Kontingent (Quota, Speichereinschränkung)
	\item Status des Benutzerkontos (gesperrt, freigeschaltet)
	\item Status des Passworts (aktuell, abgelaufen)
\end{itemize}
\lstset{caption={Erstellen eines Users},label=DescriptiveLabel}
\begin{lstlisting}
	CREATE USER <benutzername> IDENTIFIED BY <passwort>
	(DEFAULT TABLESPACE <tablespace>);
\end{lstlisting}
\lstset{caption={Ändern eines Users},label=DescriptiveLabel}
\begin{lstlisting}
	ALTER USER <benutzername> IDENTIFIED BY <passwort>;
\end{lstlisting}
\lstset{caption={Löschen eines Users},label=DescriptiveLabel}
\begin{lstlisting}
	DROP USER <benutzername> CASCADE; --CASCADE: Rechte, etc auch loeschen
\end{lstlisting}

\begin{tabular}{ll}
 \parbox{7cm}{
 \textbf{Systemrechte:}
\begin{itemize}
	\item Create Table
	\item UNLIMITED TABLESPACE
	\item CREATE USER
	\item CREATE INDEX
	\item DROP ANY VIEW
	\item ALTER USER
	\item CREATE SESSION
\end{itemize}}
 &
 \parbox{10cm}{
 \textbf{Objecktrechte für Tabellen:}
\begin{itemize}
	\item ALTER
	\item DELETE
	\item UPDATE
	\item SELECT
	\item INDEX
	\item REFERENCE
	\item INSERT
	\item EXECUTE
\end{itemize}}
\end{tabular}

\lstset{caption={Tabellen anlegen},label=DescriptiveLabel}
\begin{lstlisting}
	GRANT CREATE SESSION TO <benutzername>;
	GRANT CREATE TABLE TO <benutzername>;
	GRANT UNLIMITED TABLESPACE TO <benutzername>;
\end{lstlisting}

\section{Rollen - Sammlung von Rechten}
\lstset{caption={Definieren von Rollen},label=DescriptiveLabel}
\begin{lstlisting}
	CREATE ROLE <rollenname> [IDENTIFIED BY <passwort> || NOT IDENTIFIED];
\end{lstlisting}
\end{document}